\documentclass[../main.tex]{subfiles}

\begin{document}

The paper~\cite{popov2019} introduces a protocol 
of low communicational complexity
which allows a set of nodes to come to a 
consensus on a value of a bit by means 
of (possibly randomized) voting
(see e.g.\ \cite{Becchetti2016,cooper2014,cooper2015, elsasser2016,fanti2019,mallmann2017}
and references therein for the vast available literature 
on this subject; we mention also the classical work on (probabilistic)
Byzantine consensus protocols,
see e.g.~\cite{AguTou12, Ben83, Bra87, CasLis02, 
FelMic89, FriMosRay04, Rab83}, where,
typically, the communicational complexity is much larger).
The distinguishing feature of the mechanism described in~\cite{popov2019}
is that a larger number of
\emph{adversarial} (or \emph{Byzantine})
 nodes is allowed, which may be a (fixed)
proportion of the total number of nodes.
Those adversarial nodes 
intend to either delay the consensus,
or break it (i.e., make at least a couple of honest nodes
come to different conclusions).
 It is shown that,
nevertheless, the protocol works with high probability
 when its parameters
 are suitably chosen, and  
some explicit estimates on the probability 
that the protocol finalizes in the consensus state
in a given time are also provided (see Theorems~2.1
and~2.4 of \cite{popov2019}).

Differently from the classical work
in this area, 
it is not required that
the consensus should be achieved, with high probability, on the initial majority
value.
Rather, 
\begin{itemize}
 \item if, initially, no 
 \emph{significant majority}\footnote{Loosely speaking,
 a significant majority is something statistically 
 different from the $50/50$ situation; for example,
 the proportion of $1$-opinion is greater than~$\phi$
 for some fixed $\phi>1/2$.} 
 of nodes prefer~$1$,
 then the final consensus will be on the value~$0$ 
with high probability;
 \item if, initially, a 
 \emph{supermajority}\footnote{Again, this is 
 a loosely defined notion; 
 a supermajority is something already \emph{close}
 to consensus, e.g., more than 90\% of all nodes 
 have the same opinion.}
 of nodes prefer~$1$,
 then the final consensus will be~$1$ with high probability.
\end{itemize}
To explain why this is relevant in cryptocurrency 
applications, consider a situation when there 
are two contradicting transactions;
for example, one of them transfers all the balance of 
address~$A_1$ to address~$A_2$, while the other
transfers all the balance of 
address~$A_1$ to address~$A_3\neq A_2$.
In the case when neither of the two transactions
is strongly preferred by the nodes of the network,
by declaring both invalid we are on the safe side.
On the other hand, it would not be a good idea to \emph{always}
declare them invalid. Indeed, if we do this, 
then a malicious actor would be able to 
exploit it in the following
way: First, he places a legitimate transaction, e.g.,\
to buy some goods from a merchant. When he receives 
the goods, he publishes a double-spending transaction
as above in the hope that \emph{both} would be
canceled by the system, and so he would effectively
receive his money back (or at least take
the money away from the merchant). To avoid this kind
of development, it would be desirable if the first
transaction (payment to the merchant) which,
by that time, have probably gained some confidence 
 from the nodes, would stay confirmed,
and only the subsequent double-spend gets canceled.

A special feature of the protocol of~\cite{popov2019} is that it makes 
use of a sequence of random numbers which are either
provided by a trusted source~\cite{nist:2019} or generated
by the nodes themselves using some
decentralized random number generating protocol 
(provided that the proportion of the adversarial nodes is not too large) 
by leveraging on cryptographic primitives such as verifiable secret sharing, 
threshold signatures or verifiable delay functions,
see e.g.~\cite{cascudo2017, popov2017, schindler2018, syta2017, boneh:2018}
It is important to observe that,
even if from time to time the adversary can get
(total or partial) control 
of the random number, this can only lead to delayed consensus,
but he cannot convince different honest nodes of different 
things, i.e., safety is not violated.
Also, it is not necessary that really \emph{all}
honest nodes agree on the same number; if most 
of them do, this is already enough for the 
protocol (that is, the specific task
of random number generation does not require
any sort of ``strong consensus'').

As described in Section~\ref{sec:sybil}, Sybil protection is implemented by \emph{mana}.
Thus, any node participating in the voting process forms its quorum by weighting 
the probability of selecting nodes proportionally to their \emph{mana} as well as allowing repetitions. 
This, in turn, allows to create a bigger quorum while minimizing the communication overhead.

We do not describe all the details of the proposed
protocol here, and refer the interested reader to~\cite{popov2019} instead.
\end{document}
