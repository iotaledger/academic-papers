\documentclass[../main.tex]{subfiles}

\begin{document}
In the introduction, we mentioned that the current IOTA main network uses the Coordinator to reach consensus and, more generally, to guarantee the security of the network. However, this centralized component should only be considered as a necessary bootstrapping mechanism, rather than a long-term solution. In this section, we first discuss the current status of the IOTA main network implemented through the IOTA reference implementation (IRI) software\footnote{\url{https://github.com/iotaledger/iri}}, and then we describe the challenges we are facing when building a \textit{Coo-less} network (i.e., a network without the Coordinator). Since the Coordinator and its milestones are currently deeply embedded in IRI, removing those dependencies not only implies comprehensive changes to the software, but also leads to new research questions.

\subsection{Current IOTA implementation}\label{sec:coo-dep}

The current IOTA main network is implemented according to the IRI software, in which the Coordinator plays an important role. In the following we describe the main tasks implemented in the current main network, not all of them strictly related to consensus:

\begin{itemize}
	
	\item \textit{Manual peering}. In order to join the Tangle, a node is required to connect to some existing nodes (\textit{peering}). The current IRI software only permits \textit{manual} peering, i.e., a node operator has to manually look for the addresses of other Tangle's nodes. Peering is fundamental to propagate transactions and to synchronize to the current status of the ledger. As for the latter, milestones are useful anchors to determine whether two nodes have fallen out of synchronization: If a node’s latest solid milestone is much older than its peers', it is probably lagging behind.
	
	\item \textit{Rate control mechanism}. In order to issue a transaction, a node must solve a cryptographic puzzle (Proof-of-Work). This is necessary to guarantee that nodes do not arbitrarily spam the network, or to avoid that they inject more transactions than the network can handle.
	
	\item \textit{Tip selection strategy}. Approving transactions is a fundamental procedure which leads to the DAG structure of the Tangle. To approve a transaction, a node must verify that no inconsistencies with respect to the ledger state are introduced. Although it is not possible to enforce which transaction to validate, the original IOTA white paper suggests a tip selection algorithm based on a random walk which: (i) Discourages lazy behavior and encourages approving fresh tips; (ii) continuously merges small branches into a single large branch, thus increasing confirmation rate; (iii) in case of conflicts, kills off all but one of the conflicting branches.% Moreover, for scalability reasons the random walks start from a few milestones in the past rather than from the genesis\footnote{The exact starting point depends on the \textit{depth} parameter given to the \href{https://docs.iota.org/docs/iri/0.1/references/api-reference\#gettransactionstoapprove}{\texttt{getTransactiosToApprove} API call.}}.
	
	\item \textit{Consensus.} The main role of milestones is to determine the consensus. The Tangle applies a simple rule: A transaction is confirmed if and only if it is referenced by a milestone. In IRI, this is reflected in the \href{https://docs.iota.org/docs/iri/0.1/references/api-reference\#getbalances}{\texttt{getBalances}} and \href{https://docs.iota.org/docs/iri/0.1/references/api-reference\#getinclusionstates}{\texttt{getInclusionStates}} API calls, which indicate how many tokens an account has and whether a transaction is confirmed, respectively.
	
	\item \textit{IRI code optimization.} Instead of computing the full ledger state starting from the genesis, an intermediate state is saved for each milestone; similarly, milestones are used in \textit{local snapshots}, i.e., the IRI pruning mechanism, which allows nodes to avoid storing older parts of the Tangle.
	
\end{itemize}

\subsection{New research challenges}\label{sec:cliri_limitations}
	The most significant modelling assumption in the IOTA white paper is the \textit{honest transaction majority} condition~\cite{bramas2019}: Specifically, to be considered valid, the white paper consensus algorithm requires that the majority of transactions \textit{always} come from honest network participants. The implication is that honest nodes may need to continuously send transactions, regardless of whether they are actually using the network or not. Furthermore, achieving this hashing majority must be expensive, otherwise it would be easy for malicious agents to buy enough hashing power and overtake the network. In addition to this incentivization problem, issuing transactions is subject to Proof-of-Work (PoW). Due to its complexity, slow nodes would be excluded from participating in the network.
	
	The above concerns directly lead to new challenges and research questions which will be investigated throughout the paper:
	
	\begin{itemize}
	    \item \textit{Peering}. We require an automatic way to connect to existing nodes. Such an autopeering mechanism generates a network topology which becomes fundamental to deal with eclipse attacks and acts as a main component in certain voting protocols (see later Cellular consensus).
	    
	    \item \textit{Sybil protection}. When node identities are introduced, it is necessary to have a mechanism that prevents the creation of counterfeit identities to gain disproportionate throughput or voting weight.
	    
		\item \textit{Rate control}. A more efficient rate control algorithm is needed to solve the following tradeoff: If the PoW difficulty is too high, then small devices (e.g., phones or sensors) would take an unreasonably long amount of time to compute it, and will therefore be unable to send transactions; on the other hand, low difficulty can favor network congestion and/or spam attacks.
		
		\item \textit{Consensus}. We need a consensus mechanism which is solid under 
		the honest node majority assumption (ensured by an appropriate
		Sybil protection mechanism) without the support of the Coordinator.
		
		\item \textit{Tip Selection}. 
		In the current proposal, the security of the system does not 
		rely solely on tip selection anymore.
		Therefore, there is now more freedom in choosing the 
		(recommended) tip selection algorithm. 
		We require it to be low in computational power, to make honest transactions approved quickly and to demotivate lazy behavior, this all while keeping a good structure for the Tangle.
	\end{itemize}
	
In next section \ref{sec:node_acc}, we will introduce the notion of node identity which is a prerequirement to the solution of the above research topics.

	
\subsubsection{GoShimmer}

These new concepts and the research results tied to this should be tested in an experimental manner, in order to proceed to the next level of implementation in a protocol. An important step, therefore, is to introduce a code-base on which experiments can be be performed and the hypotheses thoroughly tested. This is achieved by implementing the concepts of the Coordicide blueprint into a prototype, called \emph{GoShimmer}\footnote{\url{https://github.com/iotaledger/goshimmer}}.

\emph{GoShimmer} is designed in a modular fashion, where each module represents one of the essential Coordicide's components as well as core components necessary to work as a full-node (e.g., gossip layer, ledger state, API). This approach enables to convert the concepts piece-by-piece and more importantly, simultaneous but independent of each other, into a prototype.
	

\end{document}
